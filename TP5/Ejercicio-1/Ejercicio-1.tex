\section{Introduction}

En el presente trabajo de laboratorio se realizaron distintas mediciones utilizando el analizador de espectros.

\section{Medición de distorsión armónica}
Como primera experiencia se midió la distorsión armónica (THD) de distintos generadores de funciones disponibles en el laboratorio. Para esto se generó una señal senoidal de $1$ Mhz y $350$ mVpp de amplitud, con el OUT TERM en HiZ, de donde la señal que llegó al analizador fue de aproximadamente $125$ mVpp. Una vez realizadas las mediciones, se compararon los valores de THD obtenidos con los calculados analíticamente y los recuperados de las hojas de datos de los fabricantes. Para el cálculo analítico, se tomó la definición de THD como:
\begin{equation}
    THD=\frac{P_1+P_2+...}{P_0+P_1+P_2+...}
    \label{eq:THD}
\end{equation}
donde $P_0$ representa la potencia del armónico fundamental, y $P_i$ la potencia del i-ésimo armónico.

\subsection{Mediciones}
Se procedió a medir la intensidad de los armónicos de la señal de entrada configurando al analizador de espectros con un Resolution Bandwith de \textcolor{red}{Rbw} y un span de \textcolor{red}{span} .

\subsection{Verificación de los resultados obtenidos}

\subsection{Conclusiones}
