\subsection{Señal Cuadrada}
Para esta sección se utilizó una señal cuadrada de $350mV$ y con un $DC$ del 50\%. Esta señal fue generada con el generador de señales Agilent.
\subsubsection{Cálculo Analítico}

\subsubsection{Simulación del Espectro}

\subsubsection{Medición}
Las mediciones se llevaron a cabo utilizando el marcador del dispositivo el cual al posicionarlo sobre el armónico que se quiere medir muestra la frecuencia y la potencia de este en $dBm$. A continuación se presentan las mediciones realizadas con el analizador de espectro:

\begin{table}[]
\scalebox{0.8}{
\begin{tabular}{@{}ccccccccccc@{}}
\toprule
$P_0$ & $P_1$ & $P_2$ & $P_3$ & $P_4$ & $P_5$ & $P_6$ & $P_7$ & $P_8$ & $P_9$ & $P_{10}$ \\ \midrule
$-11.2dBm$ & $-63dBm$ & $-18.2dBm$ & $-62.8dBm$ & $-22.6dBm$ & $63.8dBm$ & $-30.2dBm$ & $-63.4dBm$ & $-27.6dBm$ & $-61.4dBm$ & $-32.6dBm$
\end{tabular}
}
\end{table}

\subsubsection{Cálculo del Duty-Cycle}	

\subsection{Señal Triangular}

\subsubsection{Cálculo Analítico}

\subsubsection{Simulación del Espectro}

\subsubsection{Medición}

\subsection{Tren de Pulsos}

\subsubsection{Cálculo Analítico}

\subsubsection{Simulación del Espectro}

\subsubsection{Medición}

\subsubsection{Cálculo del Duty-Cycle}

\subsection{Conclusiones}