\section{Análisis de señales}

\subsection{Señal Cuadrada}

Para esta sección se utilizó una señal cuadrada de $350mV$ y con un $DC$ del 50\%. Esta señal fue generada con el generador de señales Agilent.

\subsubsection{Cálculo Analítico}

Sea una señal cuadrada $f(t)$ de amplitud $A$ y frecuencia $f_0$, el desarrollo en serie de Fourier para esta señal es:

\begin{equation}
    f(t)=\frac{A}{2}+\sum_{n=1}^{\infty} \frac{2A}{n\pi}sin(2\pi n f_0t)
    \label{eq:fouriercuadrada}
\end{equation}

Luego, $X_n=\frac{2A}{n\pi}$ representan los coeficientes de la serie de Fourier de la onda cuadrada para $n>1$ Notemos, además, que para múltiplos pares de la frecuencia fundamental la señal se anula. Luego se verán únicamente armónicos impares en el espectro.

\subsubsection{Simulación del Espectro}

Se realizó una simulación del espectro de una señal cuadrada con frecuencia $1 MHz$ y amplitud $350 mV_{pp}$. En la figura \ref{fig:simcuad} se puede observar el resultado de dicha simulación. Nótese que efectivamente no hay presentes armónicos pares.

\begin{figure}[H]
	\centering
	\includegraphics[width=0.9\textwidth]{/ImagenesEjercicio2/FFT-Cuadrada.png}
\caption{Simulación del espectro de una señal cuadrada.}
	\label{fig:simcuad}
\end{figure}


\subsubsection{Medición}
Las mediciones se llevaron a cabo utilizando el marcador del dispositivo el cual al posicionarlo sobre el armónico que se quiere medir muestra la frecuencia y la potencia de este en $dBm$. A continuación se presentan las mediciones realizadas con el analizador de espectro:

\begin{table}[]
\scalebox{0.8}{
\begin{tabular}{@{}ccccccccccc@{}}
\toprule
$P_0$ & $P_1$ & $P_2$ & $P_3$ & $P_4$ & $P_5$ & $P_6$ & $P_7$ & $P_8$ & $P_9$ & $P_{10}$ \\ \midrule
$-11.2dBm$ & $-63dBm$ & $-18.2dBm$ & $-62.8dBm$ & $-22.6dBm$ & $63.8dBm$ & $-30.2dBm$ & $-63.4dBm$ & $-27.6dBm$ & $-61.4dBm$ & $-32.6dBm$
\end{tabular}
}
\end{table}

\subsubsection{Cálculo del Duty-Cycle}	

Se entiende por duty cycle o ciclo de trabajo a la relación entre el tiempo en que la señal está activa y el período de dicha señal. Expresado en porcentaje, puede variar desde $0\%$ hasta $100\%$. Matemáticamente, para una señal cuadrada el ciclo de trabajo $D$ puede expresarse como $D=\frac{\tau}{T}.100\%$, donde $\tau$ es el tiempo en que la señal está activa y $T$ es el período.

Luego, podemos escribir el pulso cuadrado con un Duty Cycle determinado por $\tau$ como 

\begin{equation}
    x(t)=A\Pi(t-\frac{\tau}{T})
\end{equation}

cuya serie de Fourier puede expresarse como 

\begin{equation}
x(t)=\sum_{n impar}\frac{2A}{\pi n} sin(\frac{\pi n\tau}{T})
\end{equation}

Luego, como los armónicos pares se anulan, tenemos $sin(\frac{\pi n\tau}{T})=0$ si $n=2k$ con $k$ entero. De ahí se deduce que $\frac{\tau}{T}=\frac{1}{2}$, de donde el Duty Cycle es del $50\%$

\subsection{Señal Triangular}

Para esta sección se generó una señal triangular con DC del 50\% utilizando el generador GW.

\subsubsection{Cálculo Analítico}

Para una señal triangular $x(t)$ con amplitud $A$ y frecuencia $f_0$, se tiene que los coeficientes de la serie de Fourier son:

\begin{equation}
    |X_n|=
    \begin{cases}
                  \frac{4A}{\pi^2 n^2}, \text{si n es impar}\\ 
                  0, \text{si n es par} \\
     \end{cases}
\end{equation}

Se observa nuevamente que los armónicos pares valen cero. 

\subsubsection{Simulación del Espectro}

Se simuló el espectro de una señal triangular. Dicho espectro puede observarse en la figura \ref{fig:simtriang}

\begin{figure}[H]
	\centering
	\includegraphics[width=0.9\textwidth]{/ImagenesEjercicio2/FFT-Triangular.png}
\caption{Simulación del espectro de una señal triangular.}
	\label{fig:simtriang}
\end{figure}


\subsubsection{Medición}

\subsection{Tren de Pulsos}

Se generó un tren de pulsos con un DC del $33,3\%$. 

\subsubsection{Cálculo Analítico}

De forma análoga a lo desarrollado en la sección de la onda cuadrada, como el DC es del 33,3\% se verifica que $sin(\frac{n\pi}{3}=0$), de donde los armónicos múltiplos de $3$ se anularán. 

Haciendo el desarrollo en serie de Fourier se encuentra que $|X_n|=\frac{A\sqrt{3}}{n\pi}$ para todo $n$ que no sea múltiplo de $3$

\subsubsection{Simulación del Espectro}

Se simuló el espectro del tren de pulsos. El resultado puede observarse en la fiugra \ref{fig:simpulso}

\begin{figure}[H]
	\centering
	\includegraphics[width=0.9\textwidth]{/ImagenesEjercicio2/FFT-Pulsos.png}
\caption{Simulación del espectro del tren de pulsos}
	\label{fig:simpulso}
\end{figure}

\subsubsection{Medición}

\subsubsection{Cálculo del Duty-Cycle}

\subsection{Conclusiones}
