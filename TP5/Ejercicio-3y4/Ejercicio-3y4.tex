\section{Análisis de señales en AM}
Se estudiaron señales moduladas en AM generadas con dos osciloscopios, de forma que la frecuencia de la portadora sea de $1MHz$ y la frecuencia de la moduladora de $100 KHz$

Sea $S_{AM}(t)$ la señal modulada, $S_p(t)$ la señal portadora y $S_M(t)$ la señal moduladora, entonces 
\begin{equation}
    S_p(t)=A_pcos(2\pi f_pt)
\label{eq:Sp}
\end{equation}

\begin{equation}
    S_M=\frac{1}{2}mA_pcos(2\pi f_Mt)
\label{eq:Sm}
\end{equation}
siendo $A_p$ la amplitud de la señal portadora, $m$ el coeficiente de modulación y $f_p$ y $f_M$ la frecuencia de la señal portadora y la señal moduladora respectivamente.

La señal modulada puede describirse como la suma de tres señales senoidales de tres frecuencias distintas, una central y otras dos que la modulan. Matemáticamente es posible describirla según la siguiente ecuación:
\begin{equation}
    S_{AM}(t)=S_p(t)(1+mS_M(t))
    \label{eq:sam2}
\end{equation}

Desarrollando,
\begin{equation}
    S_{AM}=\frac{1}{2}A_pcos(2\pi f_pt)+\frac{1}{2}mA_p(cos(2\pi (f_p-f_M)t)+cos(2\pi (f_p+f_M)t))
    \label{eq:Sam}
\end{equation}
