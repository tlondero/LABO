\documentclass[a4paper]{article}
\usepackage[utf8]{inputenc}
\usepackage[spanish, es-tabla, es-noshorthands]{babel}
\usepackage[table,xcdraw]{xcolor}
\usepackage[a4paper, footnotesep = 1cm, width=18cm, left=2cm, top=2.5cm, height=25cm, textwidth=18cm, textheight=25cm]{geometry}
%\geometry{showframe}

\usepackage{amsmath}
\usepackage{amsfonts}
\usepackage{amssymb}
\usepackage{float}
\usepackage{graphicx}
\usepackage{caption}
\usepackage{subcaption}
\usepackage{multicol}
\usepackage{multirow}
\setlength{\doublerulesep}{\arrayrulewidth}

\graphicspath{{../Ejercicio-1/}{../Ejercicio-2/}{../Ejercicio-3y4/}{../Ejercicio-5-6y7/}{../Ejercicio-8/}}

\usepackage{hyperref}
\hypersetup{
    colorlinks=true,
    linkcolor=blue,
    filecolor=magenta,      
    urlcolor=blue,
    citecolor=blue,    
}
\newcommand\underrel[2]{\mathrel{\mathop{#2}\limits_{#1}}}
\newcommand{\quotes}[1]{``#1''}
\usepackage{array}
\newcolumntype{C}[1]{>{\centering\let\newline\\\arraybackslash\hspace{0pt}}m{#1}}
\usepackage[american,oldvoltagedirection,siunitx]{circuitikz}
\usepackage{fancyhdr}
\usepackage{units}
\usepackage{booktabs}

\usepackage{tikz}
\usetikzlibrary{babel}

\pagestyle{fancy}
\fancyhf{}
\lhead{22.42 Laboratorio de Electrónica}
\rhead{Bertachini, Lambertucci, Londero Bonaparte, Mechoulam, Scapolla}
\rfoot{\center \thepage}

\begin{document}
\section{Puente de Wien - Medición de frecuencias}

\subsection{Introducción}
En esta sección, se procederá a diseñar un puente de Wien, analizando las sensibilidades para todo el rango de medición. \par
Dicho puente permitirá conocer la frecuencia de una fuente desconocida, esto se debe a un procedimiento por el cual se comparan distintas magnitudes presentes en las ramas del puente, se puede considerar como un proceso de medición indirecta. Existen limitaciones de tipo constructivas para el rango de frecuencias que se pueden medir. El rango de frecuencias será de $10KHz$ a $200KHz$.\par
Un puente se considera en equilibrio cuando el cociente entre la tensión en la salida del puente $V_D$ y la tensión del generador $V_G$ es nulo y cuando el cociente entre una impedancia y su opuesta es equivalente al otro par dentro del circuito. La primer hipótesis nunca se podrá confirmar ya que es una afirmación de carácter netamente teórico, en la práctica nunca se obtendrá una tensión nula debido tanto a la incerteza en los métodos de medición utilizados, mostrado en el ejercicio 1, así como también a las sensibilidades propias de los componentes. \par
El puente de Wien cuenta con cuatro ramas, el primer par de ramas adyacentes cuentan con componentes puramente resistivos, el otro par está formado por circuitos RC, uno en paralelo y otro en serie, como se puede apreciar en el circuito (\ref{fig:Puente_de_wien}). \par
Las relaciones obtenidas para el mismo son las siguientes: \par
\begin{equation}
\frac{R_1}{R_3}+\frac{C_3}{C_1}=\frac{R_2}{R_4}
\end{equation}
\begin{equation}
w=\frac{1}{\sqrt{C_1C_3R_1R_3}}
\end{equation} \par
Generalmente, y esa es la manera en la que se procederá en este trabajo, el diseño del puente se realiza considerando las frecuencias desconocidas de las fuentes a medir. Se fijarán valores equivalentes para los capacitores, lo mismo sucederá con sus resistencias asociadas. 
\begin{equation}
R_1=R_3=R \quad	\wedge \quad C_1=C_3=C
\end{equation}
Dando como resultado, la siguiente relación para las ramas puramente resistivas:
\begin{equation}
R_2=2R_4
\end{equation}
Reduciendose el cálculo de la frecuencia a:
\begin{equation}
f=\frac{1}{2\pi RC}
\label{frec}
\end{equation}
Las resistencias relacionadas con los capacitores son las que serán variadas para definir el comportamiente del puente, las mismas tendrán una sensibilidad asociada. Con dicha finalidad, se utilizarán resistencias variables de precisión (preset multivueltas de 25 vueltas).

%Puente_de_wien
\begin{figure}[H]
\centering
\includegraphics[scale=0.7]{Circuitos/Puente_de_Wien.pdf}
\caption{Puente de Wien planteado}
\label{fig:Puente_de_wien}
\end{figure}
%~Puente_de_wien
Partiendo de la fórmula para la sensibilidad dada por $\Delta V_D=V_GF\delta_{Z_i}=V_GF\frac{\Delta Z_i}{Z_i}$ siendo $i$ la i-ésima rama, se obtendrán las siguientes sensibilidades para las resistencias variables $R_3$ y $R_1$.
\begin{equation}
\Delta V_{D_{R_1}}=V_g\left|\frac{\$C_1R_1}{\$C_1R_1+1}\right|\frac{\Delta R_1}{R_1}
\end{equation}
\begin{equation}
\Delta V_{D_{R_3}}=V_g\left|\$C_3R_3+1\right|\frac{\Delta R_3}{R_3}
\end{equation}
Siendo $F$ el factor cabeza de puente dado por:
\begin{equation}
F=\frac{A}{1+2A\cos(\theta_A)+A^2} \quad \wedge \quad A=\frac{Z_4}{Z_2}
\label{cabeza_de_puente}
\end{equation} 
\subsection{Desarrollo}
Primeramente, como primera decisión de diseño se decide implementar dos presets por cada ressitencia variable requerida, esto se debe a que uno servirá para ajuste grueso y el otro para un ajuste fino, logrando de esta manera obtener un ajuste más preciso que permita reducir la sensibilidad. \par
Se decide utilizar un capacitor multicapa de $1nf$. Para dicho componente, considerando las frecuencias planteadas anteriormente($10KHz$ y $200KHz$) que serán nuestras cotas, y utilizando la ecuación (\ref{frec}), se necesitan valores de resistencias de $15K915 \Omega$ y $795 \Omega$ respectivamente. 
Para el primer caso, serán utilizados dos presets de $\Omega$ y $\Omega$, por otro lado, presets de $\Omega$ y $\Omega$ serán utilizados para el segundo. \par
Luego de investigar y simular para distintos valores, se decide tomar los siguientes valores para las ramas puramente resistivasdel puente, $R_4=1K\Omega$ y $R_2=2K\Omega$. Volviendo a la ecuación (\ref{cabeza_de_puente}), obtendremos un valor cabeza de puente $A=\frac{1}{2}$, que dará un $F=0.22$.


\subsection{Conclusiones}
\end{document}