\begin{center}
\rule{\textwidth}{1pt}
\textsc{Medidor de Capacidad LABG3 \textsuperscript{\textregistered}}
\rule{\textwidth}{1pt}
\end{center}

\begin{multicols}{2}

\begin{enumerate}
	\item[1] \textbf{Características}
	\begin{itemize}
		\item Sistema de medición capacidad y resistencia serie.
		\item Rango de medición entre $100 \ nF$ y $1 \ \mu F$.
		\item Sistema de medición multivueltas.
		\item Pines hembras que permiten modificar con facilidad componentes pertinentes.
	\end{itemize}
	
	\item[2] \textbf{Descripción}\\
		El \textsc{Medidor de Capacidad LABG3~\textsuperscript{\textregistered}} es un puente que permite medir capacitancias en un rango de $100 \ nF \sim 1 \ \mu F$, con un error máximo del 15 \%. Este cuenta con un sistema de conexión de componentes que permite intercambiarlos con suma facilidad, para luego extraer valores de interés de estos, a partir de sus mediciones.
	
	\item[3] \textbf{Modo de uso}\\
	Colocan los presets y el capacitor a medir en los pines hembra, considerando la polarización de este último en caso de ser necesario. Alimentar el circuito con una tensión acorde (seleccionada con criterio). Medir la tensión entre los cables rojo y negro, calibrando mediante los presets de forma tal que se consiga que dicha tensión sea lo más próxima a cero. Finalmente, retirar los prestes y medir estos para conseguir los datos necesarios.
		
	\item[4] \textbf{Recomendaciones}
		\begin{itemize}
		\item Emplear amplificador de instrumentación en la tensión de salida para realizar mediciones de mayor precisión.
		\item Utilizar tensiones de entradas mucho mayores al piso de ruidos.
	\end{itemize}

\end{enumerate}
\end{multicols}