\documentclass[a4paper]{article}
\usepackage[utf8]{inputenc}
\usepackage[spanish, es-tabla, es-noshorthands]{babel}
\usepackage[table,xcdraw]{xcolor}
\usepackage[a4paper, footnotesep = 1cm, width=18cm, left=2cm, top=2.5cm, height=25cm, textwidth=18cm, textheight=25cm]{geometry}
%\geometry{showframe}

\usepackage{amsmath}
\usepackage{amsfonts}
\usepackage{amssymb}
\usepackage{float}
\usepackage{graphicx}
\usepackage{caption}
\usepackage{subcaption}
\usepackage{multicol}
\usepackage{multirow}
\setlength{\doublerulesep}{\arrayrulewidth}

\graphicspath{{../Ejercicio-1/}{../Ejercicio-2/}{../Ejercicio-3y4/}{../Ejercicio-5-6y7/}{../Ejercicio-8/}}

\usepackage{hyperref}
\hypersetup{
    colorlinks=true,
    linkcolor=blue,
    filecolor=magenta,      
    urlcolor=blue,
    citecolor=blue,    
}
\newcommand\underrel[2]{\mathrel{\mathop{#2}\limits_{#1}}}
\newcommand{\quotes}[1]{``#1''}
\usepackage{array}
\newcolumntype{C}[1]{>{\centering\let\newline\\\arraybackslash\hspace{0pt}}m{#1}}
\usepackage[american,oldvoltagedirection,siunitx]{circuitikz}
\usepackage{fancyhdr}
\usepackage{units}
\usepackage{booktabs}

\usepackage{tikz}
\usetikzlibrary{babel}

\pagestyle{fancy}
\fancyhf{}
\lhead{22.42 Laboratorio de Electrónica}
\rhead{Bertachini, Lambertucci, Londero Bonaparte, Mechoulam, Scapolla}
\rfoot{\center \thepage}

\begin{document}

Para la siguiente sección, se buscó diseñar un puente que permita medir capacitores desde $100 \ nF$ hasta $1 \ \mu F$, trabajando a una frecuencia de $20 \ kHz$. Los capacitores a medir con dicho instrumento se caracterizan por poseer un factor de disipación D en un rango acotado entre $0.02$ y $0.12$.

Con lo dicho anteriormente, se tuvo que elegir entre tres posibles puentes: el serie, el paralelo y el de Schearing. Debido a que este último se emplea para capacitores con de muy bajas perdidas, es decir, con un $D \ < \ 10^{-3}$, mientras que el paralelo se utiliza para capacitores de altas perdidas, se optó por valerse de un puente serie, ya que este está destinado a ser empleado para capacitores con un rango de perdidas similar al que requiere.

\begin{figure}[H]
\begin{center}
\begin{circuitikz}[european voltages]
	\draw (0,0) to[voltmeter, label=$V_d$] ++(3,0) to[R, l=$R_4$] ++(0,-2) to[short] ++(-4,0) node[ocirc](-vg){};
	\draw (0,0) to[R, l=$R_3$] ++(0,-2);
	\draw (0,0) to[R, l_=$R_1$] ++(0,1.5) to[C, l_=$C_1$] ++(0,1.5) to[short] ++(3,0) to[C, l=$C_x$] ++(0,-1.5) to[R, l =$R_x$] ++(0,-1.5);
	\draw (0,3) to[short] ++(-1,0) node[ocirc](vg){};
	\draw (-vg) to[open, v^= $V_g$] (vg);
\end{circuitikz}
	\caption{Puente serie implementado}
	\label{fig:puenteserie}
\end{center}
\end{figure}

Luego, mediante el análisis de sensibilidades, considerando los valores máximos y mínimos de capacitores y estableciendo $C_1 = 1 \ nF$ y $R_1 = 100 \Omega$, se obtienen los siguientes valores para los demás componentes:

\begin{equation*}
\begin{split}
	R_{1_{Min}} =& \ \frac{5 \cdot 10^{-7}}{\pi C_1} = 159 \ \Omega \\
	R_{1_{Max}} =& \ \frac{3 \cdot 10^{-6}}{\pi C_1} = 1 \ k\Omega \\
	R_{3_{Min}} =& \ \frac{R_4 \cdot 10^{-7}}{\pi C_1} = 10 \ k\Omega \\
	R_{3_{Max}} =& \ \frac{R_4 \cdot 10^{-6}}{\pi C_1} = 100 \ k\Omega
\end{split}
\end{equation*}
 
\end{document}